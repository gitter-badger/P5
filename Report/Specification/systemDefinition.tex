\section{System Definition}\label{sec:systemDefinition}

This section formally describes in which context the system is to be used, its expected functions, the philosophy of the system, the conditions in which the system is used, the technical aspect of the system, and the objects the system should recognise and process.

The definition of the system can be decomposed into functionality, application domain, conditions, technology, objects, and responsibility. This method of decomposing the definition of the system is taken from \cite{mathiassen2001objektorienteret}.

\paragraph{Functionality}
The system should be able to learn usage patterns of users, and repeat them autonomously, still taking into consideration the possibility of temporary exceptions to daily routines.

\paragraph{Application Domain}
The system is meant to be used in homes where one or more users live. Each user may have different usage patterns and sometimes irregular actions not part of their daily routine.

\paragraph{Conditions}
The system should be usable for as many different people as possible. In addition, it should do so without conflicting with the behaviour of the user. User irritations caused by the system should be kept to a minimum.

\paragraph{Technology}
The system should have small and cheap devices with accommodating sensors and actuators scattered strategically placed inside the home. These devices should be energy efficient in their use. Another computational device should analyse sensor data collected by the small devices and, based on the analysis, create behavioural rules for the devices to follow and create an autonomously automated system.

\paragraph{Objects}
The system should recognise the users' patterns and their interaction with the rest of the home.

\paragraph{Responsibility}
The system should be as invisible as possible, only creating behaviours that are aligned with the users wishes. Meaning that the system should preferably create rules from truly emerging behaviours, not exceptional ones, thereby avoiding to annoy the user. This is done to minimise friction between the system and the user. This means that the user should not be frustrated about the actions performed by the system. To achieve this the system should be conservative about its actions as to not perform actions the user would otherwise not have made.

\subsection{Summary}
The system should ease the lives of its users by performing actions autonomously on their behalf. These actions should be learned by monitoring users' usage patterns throughout the day. As the user may perform irregular actions, the system should be conservative about performing actions. The systems should therefore only perform actions when it has a understanding of the action pattern. The system should be as invisible to the user as possible. It should be an aid, not an element of frustration. The system should use small computational devices, that conserves energy, to monitor the usage patterns of the users. The data acquired in this way should be sent to another computational device that analyses it, to further improve the knowledge of the users' behavioural patterns, thereby improving the quality of actions the system can perform.

Many sensors working together to learn users’ usage patterns, e.g. everyday at about 7.30, sensor 1 is activated and then the user turns the light on. This could be learned by a machine so that the next time sensor 1 is activated at about 7.30, the system automatically turns on the light.
