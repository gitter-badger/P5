\subsubsection{Experiments}
In order to find out relevant properties of the two photoresistors, some experiments were conducted. The experiments were conducted in order to find out the following properties:

\begin{itemize}
  \item Precision of sensors
  \item Ideal configuration of resistors
\end{itemize}

\paragraph{Precision}\label{subsub:precision}
To test the precision of the photoresistors, an experiment was designed.
\subparagraph{Hypothesis}
It was expected that the photoresistors would exhibit some imprecision, as they were cheap, amateur-grade resistors.
\subparagraph{Test Procedure}
To test the precision of the photoresistors, they were put through three different tests. In each test, the resistors were exposed to constant light. Under these constant light conditions, many readings were taken from the resistors, in order to find the variance which they exhibited.
The tests varied in the amount of light exposed to the resistors. The light levels for the tests were as follows:

\begin{description}
  \item[Minimal light]
  This scenario was constructed by placing the resistors under a staircase, in a dimly lit room.
  \item[Average room light]
  This scenario was constructed by placing the resistors in a small, unoccupied room with no windows, with a single light bulb lit.
  \item[Intense light]
  This scenario was constructed by aiming a flashlight directly at the resistors at point blank range.
\end{description}

To record the readings from the resistors, the code in \cref{lst:arduinoPhotoCode} was used for the Arduino. This program simply reads the value from the two sensors and prints it to the serial output, which in this case was a connected computer.

\lstset{language=C}
\begin{lstlisting}[label = lst:arduinoPhotoCode, caption = Arduino program for photoresistor tests]
#define sensor1Pin 0
#define sensor2Pin 1
void setup() {
  Serial.begin(9600);
}

void loop() {
  int val1 = analogRead(sensor1Pin);
  int val2 = analogRead(sensor2Pin);
  Serial.print(val1);
  Serial.print(",");
  Serial.print(val2);
  Serial.print("\n");
}
\end{lstlisting}

This program was allowed to run long enough in each test that over 1000 readings were taken in each scenario.
To find the minimum, maximum and average values, as well as the difference between maximum and minimum in percent, the data was copied into a file, which was then input to the C\#-program seen in \cref{lst:cShPhotoCode}.

\lstset{language=[Sharp]C}
\begin{lstlisting}[label = lst:cShPhotoCode, caption = C\# data processing code]
using System;
using System.IO;

namespace ArduinoPhotoTest {
  class Program {
    [STAThreadAttribute]
    static void Main(string[] args) {
      StreamReader file = new StreamReader(@"Path/To/File.txt");
      string line;
      int a0Min = 99999;
      int a1Min = 99999;
      int a0Avg = 0;
      int a1Avg = 0;
      int a0Max = 0;
      int a1Max = 0;
      int counter = 0;
      while ((line = file.ReadLine()) != null) {
        int a0 = Convert.ToInt32(line.Split(',')[0]);
        int a1 = Convert.ToInt32(line.Split(',')[1]);
        a0Avg += a0;
        a1Avg += a1;
        if (a0Min > a0) a0Min = a0;
        else if (a0Max < a0) a0Max = a0;
        if (a1Min > a1) a1Min = a1;
        else if (a1Max < a1) a1Max = a1;
        counter++;
      }
      a0Avg /= counter;
      a1Avg /= counter;
      double a0Diff = ((double)a0Max / (double)a0Min) * 100d - 100;
      double a1Diff = ((double)a1Max / (double)a1Min) * 100d - 100;
      string text = string.Format("a0Avg = {0}\na0Min = {1}
                                  \na0Max = {2}\na0Diff = {6}
                                  %\na1Avg = {3}\na1Min = {4}
                                  \na1Max = {5}\na1Diff = {7}%",
        a0Avg, a0Min, a0Max, a1Avg, a1Min, a1Max, a0Diff, a1Diff);
      Console.WriteLine(text);
      System.Windows.Forms.Clipboard.SetText(text);
      Console.Read();
    }
  }
}
\end{lstlisting}
\subparagraph{Results}
The results of the tests can be seen in \cref{tab:precisionTestResults}. "S1" refers to the first photoresistor. and "S2" refers to the second.

  \begin{table}
\rowcolors{1}{white}{lightgray}
    \centering
    \begin{tabular}[H!]{m{4.5em} c c c c c c c c}
      Scenario & S1 min & S1 max & S1 avg & S1 diff & S2 min & S2 max & S2 avg & S2 diff \\
      \hline
      Minimum light & 136 & 138 & 137 & 1,47 \% & 25 & 27 & 26 & 8.00 \% \\
      Average room light & 875 & 889 & 883 & 1,60 \%  & 634 & 675 & 656 & 6,47 \% \\
      Intense light & 931 & 933 & 932 & 0,21 \%  & 772 & 775 & 773 & 0,39 \% \\
    \end{tabular}
    \caption{Precision test results}\label{tab:precisionTestResults}
  \end{table}

\subparagraph{Partial Conclusion}
As expected, the resistors did exhibit some imprecision under constant light. S2, however, was much worse than S1 under the minimum and average light scenarios, having more than 5 times the difference between minimum and maximum values. Both S1 and S2 were very consistent under intense light.

\paragraph{Configuration}
In order fully take advantage of the two photoresistors, an experiment was devised to find a configuration of pull-down resistors, which would offer a large range of meaningful values, i.e. be able to detect subtle differences under low light as well as intense light scenarios.
\subparagraph{Hypothesis}
Using different pull-down resistors on the photoresistors would make it possible to extend the range of values, which the sensor pair could register.
\subparagraph{Test Procedure}
The scenarios described in \cref{subsub:precision} were used in this experiment as well. Between each set of scenarios, a pull-down resistor value was changed. As the 10K $\Omega$ resistor used in the previous experiments responded well to low light conditions, and was almost impossible to bottom out, another resistor-value was sought which could complement this. To increase the tolerance to light, available resistors with less resistance than 10K $\Omega$ were investigated.
\subparagraph{Results}
The results from the configuration experiments can be seen in \cref{tab:1KTestResults,tab:2.2KTestResults,tab:4.4KTestResults,tab:10KTestResults,tab:12.2KTestResults}.

\begin{table}[H]
\rowcolors{1}{white}{lightgray}
  \centering
  \begin{tabular}{l c c c c}
    Scenario & 1K $\Omega$ min & 1K $\Omega$ max & 1K $\Omega$ avg & 1K $\Omega$ diff \\
    \hline
    Minimum light & 0 & 0 & 0 & 0.00 \% \\
    Average room light & 16 & 18 & 17 & 12.50 \% \\
    Intense light & 406 & 408 & 406 & 0.49 \% \\
  \end{tabular}
  \caption{1K $\Omega$ resistor results}\label{tab:1KTestResults}
\end{table}

\begin{table}[H]
\rowcolors{1}{white}{lightgray}
  \centering
  \begin{tabular}{l c c c c}
    Scenario & 1K $\Omega$ min & 2.2K $\Omega$ max & 2.2K $\Omega$ avg & 2.2K $\Omega$ diff \\
    \hline
    Minimum light & 0 & 0 & 0 & 0.00 \% \\
    Average room light & 49 & 58 & 53 & 18.37 \% \\
    Intense light & 609 & 612 & 610 & 0.49 \% \\
  \end{tabular}
  \caption{2.2K $\Omega$ resistor results}\label{tab:2.2KTestResults}
\end{table}

\begin{table}[H]
\rowcolors{1}{white}{lightgray}
  \centering
  \begin{tabular}{l c c c c}
    Scenario & 4.4K $\Omega$ min & 4.4K $\Omega$ max & 4.4K $\Omega$ avg & 4.4K $\Omega$ diff \\
    \hline
    Minimum light & 7 & 11 & 11 & 57.14 \% \\
    Average room light & 282 & 287 & 284 & 1.77 \% \\
    Intense light & 893 & 895 & 894 & 0.22 \% \\
  \end{tabular}
  \caption{4.4K $\Omega$ resistor results}\label{tab:4.4KTestResults}
\end{table}

\begin{table}[H]
\rowcolors{1}{white}{lightgray}
  \centering
  \begin{tabular}{l c c c c}
    Scenario & 10K $\Omega$ min & 10K $\Omega$ max & 10K $\Omega$ avg & 10K $\Omega$ diff \\
    \hline
    Minimum light & 136 & 138 & 137 & 1.47 \% \\
    Average room light & 875 & 889 & 883 & 1.60 \% \\
    Intense light & 931 & 933 & 932 & 0.21 \% \\
  \end{tabular}
  \caption {10K $\Omega$ resistor results}\label{tab:10KTestResults}
\end{table}

\begin{table}[H]
  \rowcolors{1}{white}{lightgray}
  \centering
  \begin{tabular}{l c c c c}
    Scenario & 12.2K $\Omega$ min & 12.2K $\Omega$ max & 12.2K $\Omega$ avg & 12.2K $\Omega$ diff \\
    \hline
    Minimum light & 44 & 52 & 51 & 18.18 \% \\
    Average room light & 592 & 622 & 606 & 5.07 \% \\
    Intense light & 979 & 982 & 980 & 0.31 \% \\
  \end{tabular}
  \caption{12.2K $\Omega$ resistor results}\label{tab:12.2KTestResults}
\end{table}

\subparagraph{Partial Conclusion}
Using a 12.2K pull-down resistor on one photoresistor and a 4.4K pull-down resistor on the other gives a very impressive range of meaningful values, where one photoresistor is very sensitive to low light conditions and very tolerant of intense light conditions, and the other is the inverse. This pairing seems more than adequate for the usage in this project.
