\subsection{System Interfaces}

Overall the system has to perform three tasks
\begin{itemize}
\item Data collection of events in the problem domain
\item Analysis of this data
\item Perform some sensible action based on the analysis
\end{itemize}
This section discusses different architectural designs
for the system.

\subsubsection{Monolithic System} One possibility is to have one central possessing unit that receives all sensor data, sanitise and analyse this, decides which, if any, action to perform, and performs it via its actuators. The system could be made in two ways.

The first solution is that the user has to add sensors and actuators to the system manually and inform the system about what they are. This goes very much against the principle of easy usability since the user has to be able to add sensors and actuators manually to the system and inform the system about them. It is also very vulnerable for user mistakes, which if made, could render the whole system useless\footnote{If the user informs that he/she has connected a sensor to the network but actually has connected another sensor or even an actuator, the system will have false assumptions about how to sanitise the data or that it does not receive data but has to use the module and will based on have a wrong model of the problem domain}. The system would furthermore also be limited by the total amount of sensors the processing unit physically can have.

The second solution is that the system is pre made with a fixed amount of sensors and actuators and the user can simply not couple more hardware on the system. This is a problem since all houses are different and needs different amount of sensors. Either we will have too many sensors in some houses, thereby spending too much power compared to what is needed or there will be too few sensors in some houses meaning that the system will not be very effective and/or non existing in some rooms. This goes against the principle of the need for system adapting to the house and not the house adapting to the system.

Neither of these solutions are very modular and both goes against the requirements described in, \cref{System Definition}\jenote{Referer til system definitionen når den er her}. Also they are both in general not very practical. Therefore another design is needed.

\paragraph{Distributed system}
A second solution tries to cope with the problem of modularity and ease of use by distributing the sensors and actuators on several units. The system is comprised of multiple subsystems that each have a processing unit that is aware of what sensors and actuators are connected to it and how to either sanitise their data or activate them. They can communicate and through combined knowledge get a complete model of the problem domain. The user can connect each subsystem to the network\footnote{This could be done easy via wireless communication so the user only has to provide the subsystem with power and not run wires in their home} and can connect exactly the amount of subsystems needed for his/her home. The problem with this solution is that the data has to be centralised or passed around in order to be analysed to perform machine learning. Furthermore the analysis is a complex task, see \cref{subsec:Functions}, and requires another kind of hardware then the sensors and actuators. To solve this in the least complex way the system was extended to the following.

\subsubsection{Seperated Subsystems} A seperation of data collection and actuation of tasks, and analysis of
data, is needed. Here a distributed system of small low-power processing units collect the sensor data and complete the actuation of tasks and the analysis of the sensor data is be performed on a bigger more powerfull computation device that is more able.

Under this seperation the system consists of two subsystems: sensor/actuator
and analysis. \sinote{Maybe also a user interfacing subsystem to view what is
going on in the system}

\subsubsection{Communication Schemes} To form the complete system, some
communication must take place between the aforementioned subsystems. There are
mulitple designs possible, but they all have some commonalities.

In every design, the sensor subsystem must send its data to the analysis
subsystem to be processed. However there can be variances in what is send from
the analysis subsystem to the actuator subsystem.

\begin{enumerate}
\item \label{enum:send_rule} Send rules that match usage patterns of the user.
The actuator subsystem is then responsible for converting the rules into
appropriate actions. For example, a rule could be to turn on a light in the
bathroom at 7.00 in the morning.
\item \label{enum:send_action} Send actions directly to be performed immediately
by the actuator/sensor subsystem. For example, turn on the light in the bathroom.
\end{enumerate}

Considering \cref{enum:send_rule} the actuator/sensor subsystem must have enough memory
to store all rules sent to it that are still valid. This either limits the
devices that can act as actuators or the number of rules that can ever be active
at the same time. In addition there is also a requirement for the processing
speed of these rules, and special component requirements for the devices in the
actuator subsystem. For example, if the rule \enquote{Turn on the light in the
bathroom at 7.00 in the morning} is to be processed, a mechanism for monitoring
time is needed. The benefits of the system is that it is very quick to act to sensor input compared to \cref{enum:send_action} since the data does not have to be communicated to the central subsystem before the actuator subsystem can act.

Considering \cref{enum:send_action} the hardware requirements for the devices in
the actuator subsystem are not as strict. As soon as a action is received from
the analysis subsystem, the action is to be performed. One disadvantage of this
design is that actions have to be sent explicitly every time actions are to be
performed and that this can be a relatively slow process. On the other hand, the design is simpler as the analysis subsystem
does not need to send messages regarding invalidation of rules to the actuator
subsystem.

\sinote{Needs a choice of architecture, discussion in group}
