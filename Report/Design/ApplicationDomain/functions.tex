\subsection{Functions}

This section explores functions the system should support to solve the requirements presented in \cref{}\sinote{referer til requirements når de er færdige}.

\label{subsec:Functions}
\begin{table}[hbtp]
\centering
\begin{tabular}{lcr}
\toprule
\textbf{Name}																& \textbf{Complexity}	& \textbf{Type} \\
\midrule
User turns lights off/on				& Simple	& Update  \\
User dims lights					& Simple	& Update  \\
User turns off/on appliances			& Simple	& Update  \\
User marks behaviour as wrong			& Simple	& Update  \\
Change in light intensity in the room		& Simple	& Update  \\
System turns off/on lights			& Simple	& Signal  \\
System dims lights				& Simple	& Signal  \\
System turns off/on appliances			& Simple	& Signal  \\
Appliance is on standby				& Medium	& Update  \\
User enters room				& Medium	& Update  \\
User leaves room				& Medium	& Update  \\
Find specific placement of user			& Complex	& Compute \\
Calculate behavioural rules for the system	& Complex	& Compute \\
\bottomrule
\end{tabular}
\caption{Function table.}
\label{table:functionlist}
\end{table}

\subsubsection{Find specific placement of user}
The system is able to register input from multiple ultra sound and infra red sensors. Using these different sensors distance to the user the system is able to find the specific placement of the user in the room.

\subsubsection{Calculate behavioural rules for the system}
The system uses it sensor inputs together with its general knowledge of the world (like weekday, date and time) to automatically analyse the behavioural patterns of the user and find out when and how the user uses his light and appliances. Like, the user always turns off his or her lights when he leaves all rooms at 07:45. This analysis sets up the rules for the actuators in the system so they can eventually perform the tasks for the user.
