\subsubsection{User Interfaces}
There are two kinds of user interaction in the system. From here on called passive and active interaction.

\paragraph{Passive User Interaction}
Passive user interaction is the interaction the user makes with the system that
should not be noticed as interaction. This is an interaction designed to be
invisible for the user and to mimic how the user already behaves in their home.

For instance, the user is used to turning on the light by pressing a switch. The
system would integrate with the switch so that it could detect the user
activity. By integrating with existing interfaces, the user will not notice the
interaction with the system.

\paragraph{Active User Interaction}
The active user interaction is the interaction happening with the user actively
trying to control or affect the system and is what normally happens through a
digital user interface such as an application on a smartphone.

\paragraph{Design of User Interface}
Passive user interaction is preferred since it adheres very well with the
requirements about the system needing to be easy to use, as stated in
\cref{}\jenote{Referer til requirements når vi får lavet dem}, while still being
able to gather information about the users patterns.

Through passive user interaction the system can get information about the
patterns of the user, but this information is not enough for the system to be
fully operational. Machine learning is relying on some way of evaluating its
behaviour in order to improve it. Therefore a method of evaluating the behaviour
of the system, from a users point of view, is needed.
 
To reduce active user interaction, the user should only inform the system when
it has made a mistake. The interface could also be able to inform the system
when it has made a correct action but since the system should be conservative in
its behaviour, meaning that it is far better to let the user keep doing the
tasks than doing something that is against the users wishes, the system will be
doing a lot more already known correct tasks and adjustments to them.

Furthermore the goal of the system is to only make right choices. Both of these
principles would mean that if the user needed to reward correct behaviour the
system would need constant or at least a lot more interaction, thereby going
against the principle of invisibility. The system will therefore assume no
interaction means correct behaviour.

The interaction from the user will be active by definition, because the user has
to be aware that he/she is informing the system that it does something wrong,
but it can draw from the ideas of the passive user interaction. Instead of
explicitly informing the system that it does something wrong through a
centralised unit or a computer on the network the user could simply reverse the
action of the user. For instance if the system turns of the light when the user
does not want it the user will naturally turn it on again. This could serve as
the active interaction and be an explicit way of informing the system of wrong
behaviour. This will be used as much as possible as the solution for active user
interaction.

Unfortunately this is not enough. For instance the user could only find out
later that the system has made a mistake, for instance if the system turns on
the light in a different room and the user only finds out 30 minutes later, the
system should not assume that the user only wants the lights on for 30 minutes
and the user needs to be able to inform it that the whole action was wrong. Here
a normal user interface is needed though which should be done in the central
unit in the system, see \cref{System Interfaces}, or a PC connected to the
system. The problem with this interaction is that it has the potential of going
against the principle of easy user interaction and caution is needed here to
make it as easy to use and needed as little as possible.

Skriv også at det er en ide at brugeren ved hvad der foregår i systemet. F.eks.
at det lært en ny ting.