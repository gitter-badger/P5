\subsubsection{System Interfaces}

Overall the system has to perform three tasks; data collection of events in the
application domain, analysis of this data, and performing some sensible action
based on the analysis. This section discusses different architectural designs
for the system.

\paragraph{Monolithic System} One possibility is for the system to at the sensor
subsystem decide which actions to perform based on sensor data. The system is
then the sensors and nothing else. However, in order to analyse the sensor data,
extracting usage patterns, the computational power needed is greater than what
low-power battery driven devices can provide. Therefore another design is
needed.

\paragraph{Seperated Subsystems} A seperation of data collection, analysis of
data, and actuation of tasks, is needed. This can be done by letting low-power
battery driven devices collect the sensor data. If these sensor-controlling
devices are not fully utilised computationally-wise, they can also act as
actuators, reacting to particular events in the application domain. The analysis
of the sensor data has to be performed on another computation device that is
more able.

Under this seperation the system consists of three subsystems: sensor, actuator
and analysis. \sinote{Maybe also a user interfacing subsystem to view what is
going on in the system}

\paragraph{Communication Schemes} To form the complete system, some
communication must take place between the aforementioned subsystems. There are
mulitple designs possible, but they all have some commonalities.

In every design, the sensor subsystem must send its data to the analysis
subsystem to be processed. However there can be variances in what is send from
the analysis subsystem to the actuator subsystem.

\begin{enumerate}
\item \label{enum:send_rule} Send rules that match usage patterns of the user.
The actuator subsystem is then responsible for converting the rules into
appropriate actions. For example, a rule could be to turn on a light in the
bathroom at 7.00 in the morning.
\item \label{enum:send_action} Send actions directly to be performed immediately
by the actuator subsystem. For example, turn on the light in the bathroom.
\end{enumerate}

Considering \cref{enum:send_rule} the actuator subsystem must have enough memory
to store all rules sent to it that are still valid. This either limits the
devices that can act as actuators or the number of rules that can ever be active
at the same time. In addition there is also a requirement for the processing
speed of these rules, and special component requirements for the devices in the
actuator subsystem. For example, if the rule \enquote{Turn on the light in the
bathroom at 7.00 in the morning} is to be processed, a mechanism for monitoring
time is needed.

Considering \cref{enum:send_action} the hardware requirements for the devices in
the actuator subsystem are not as strict. As soon as a action is received from
the analysis subsystem, the action is to be performed. One disadvantage of this
design is that actions have to be sent explicitely every time actions are to be
performed. On the other hand, the design is simpler as the analysis subsystem
does not need to send messages regarding invalidation of rules to the actuator
subsystem.

\sinote{Needs a choice of architecture, discussion in group}