\subsubsection{System Interfaces}

This section discusses different architecture designs for the system.

\paragraph{Monolithic System}
One possibility is for the system to at the sensor subsystem decide which
actions to perform based on sensor data. The system is then the sensors and
nothing else. However, in order to analyse the sensor data, extracting usage
patterns, the computational power needed is greater than what low-power battery
driven devices can provide. Therefore another design is needed.

\paragraph{Seperated Subsystems}
A seperation of data collection, analysis of data, and actuation of tasks, is
needed. This can be done by letting low-power battery driven devices collect the
sensor data. If these sensor-controlling devices are not fully utilised
computationally-wise, they can also act as actuators, reacting to particular
events in the application domain. The analysis of the sensor data has to be
performed on another computation device that is more able.

Under this seperation the system consists of three subsystems: sensor, actuator
and analysis. \sinote{Maybe also a user interfacing subsystem to view what is
  going on in the system}

\paragraph{Communication Schemes}
To form the complete system, some communication must take place
between the aforementioned subsystems. There are mulitple designs possible, but
they all have some commonalities.

In every design, the sensor subsystem must send its data to the analysis
subsystem to be processed. However there can be variances in what is send from
the analysis subsystem to the actuator subsystem.

\begin{enumerate}
\item Send rules that match usage patterns of the user. The actuator subsystem
  is then responsible for converting the rules into appropriate actions. For example, a rule
  could be to turn on a light in the bathroom at 7.00 in the morning.
\item Send actions directly to be performed immediately by the actuator
  subsystem. For example, turn on the light in the bathroom.
\end{enumerate}
