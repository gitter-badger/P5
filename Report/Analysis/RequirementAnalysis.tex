\section{Interview Reports}
To get a better understanding of what users wants from a home automation system we looked at several different reports. One \cite{HAInterviews} which did interviews regarding general home automation and one \cite{AdaptiveInterviews} which did interviews specifically regarding the smart thermostat Nest, with focus on machine intelligence integrated as part of a home automation solution. 
\\\\
\cite{HAInterviews} makes an interesting distinction between two terms; Smart Home and Home Automation. A Smart Home implies that the home "adapts to the inhabitants" and Home Automation at its core is simply grouping multiple actions together and binding them to one user action (such as turning every light on in a room using only one button).\ranote{Think its a relevant distinction, might be misplaced though.}
\\\\
Two significant barriers to entry into home automation are inflexibility and poor manageability\cite{HAInterviews}. Inflexibility is concerned with the idea that installing a home automation system requires changing wires often in the walls or even remodelling parts of the house to accommodate more extensive home automation solutions.  Because of this having the individual modules connect to the, possible, central server is crucial.
Manageability describes the maintainability of the system along with the ability to change how the system behaves. In \cite{HAInterviews} they noted that even though their participants were well equipped to handle such systems they still struggled and this would be a problem for widespread adoption. Especially the initial period proved to be very iterative because when the systems the users had bought were actually implemented it turned out they didn't want or need part of the functionality. Especially functions tied to a schedule were problematic because people realised their lives weren't as structured as they had thought\footnote{One person said: "I'd wake up and music is playing in my bathroom(...)all these Jetson type things. And the challenge with that (...) I don't live that structured of a life(...)}. Together with this problem users found that creating general rules were difficult, it's the "challenge of interference in the presence of ambiguity" \cite{HAInterviews}. It is difficult for a system, almost impossible for a system without machine learning, to infer the situation based on hard rules if there is any ambiguity. The fact that a lot of iteration were required also meant that the user needed to interact with the system often. This means that the user interface were important.
\\\\
Looking at some of the problems outlined above a solution could be to incorporate machine intelligence or machine learning. Which is what the Nest thermostat\cite{AdaptiveInterviews} tried to do. Although as outlined in \cite{AdaptiveInterviews}, with the version of Nest they used, Nest also had a difficult time distinguishing between routine behaviour and temporary adjustments. The machine intelligence integrated in Nest were used for a function called "Auto-Schedule". Which would try to learn the user's schedule and have the thermostat react according to that. But the machine learning were, as one user described it\cite{AdaptiveInterviews}, too "aggressive". E.g. when he changed the temperature once, the system would assume he would always want it like that and adjust the schedule.
Even though the machine intelligence part of the Nest system explored in \cite{AdaptiveInterviews} arguably failed (as multiple users either turned it off completely or only used it as part of the study), it is easy to see that a stronger system could solve some of the problems presented in \cite{HAInterviews}. 
\subsection{Temporary, for geovanni}
Based on this section we wanted to conclude that we now have multiple functional requirements;
The user should not need to interact with the system
The system should be entirely wireless
The system should be able to recognize whether an action is a temporary adjustment or a change in schedule.
