%mainfile: ../master.tex
\section{Conclusion \& Requirements}\label{sec:requirements}
This section will conclude on the analysis and from the knowledge gathered define the requirements of the system.

\subsection{Requirements}
The system should ease the lives of its users by performing actions autonomously
on their behalf. These actions should be learned by monitoring users' usage
patterns throughout the day. As the user may perform irregular actions, the
system should be conservative about performing actions. The systems should
therefore only perform actions when it has a understanding of the action
pattern. The system should be as invisible to the user as possible. The system
should be an aid, not an element of frustration. The system should use small
computational devices, that save energy consumption, to monitor the usage
patterns of the users. The data acquired in this way should be sent to another
computational device that analyses it, to further improve the knowledge of the
users' behavioural patterns, thereby improving the quality of actions the system can perform.

\cref{int:requirement} Based on these two reports and previous analysis our system should be entirely wireless, be able to recognise whether an action is a temporary adjustment or a change in schedule, and minimise user interaction while still giving the user enough information to interact with the system appropriately.

The system should focus on creating a smart home system, see \cref{sec:interviewReports}, that focus on lighting and electronic appliances. It should do so by utilising machine intelligence with machine learning in a manner that does not disturb the daily routines of the user by
\begin{itemize}
  \item Being conservative, see \cref{sec:interviewReports}, meaning that it is better not to act then to act wrongly
  \item Being reactive, meaning that the user should not notice any delay from the initiation of events to the execution
\end{itemize}

The setup and usage of the system should focus on fulfilling the following three criteria in relation to the user, found from \cref{sub:existingSystems}
\begin{itemize}
  \item Cheap
  \item User friendly
  \item Personalised
  \item Flexible
\end{itemize}

From section \cref{sub:Technologies} we have that the system must have sensor variety that can cover
\begin{itemize}
  \item Time
  \item Light intensity
  \item Location and spacial awareness
  \item Awareness of active devices
\end{itemize}
