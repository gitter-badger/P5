%mainfile: ../master.tex
\section{Conclusion \& Requirements}\label{sec:requirements}
This section will conclude on the analysis and from the knowledge gathered define the requirements of the system.

\cref{int:requirements} Based on these two reports and previous analysis our system should be entirely wireless, be able to recognise whether an action is a temporary adjustment or a change in schedule, and minimise user interaction while still giving the user enough information to interact with the system appropriately.

The system should focus on creating a smart home system, see \cref{sec:interviewReports}, that focus on lighting and electronic appliances. It should do so by utilising machine intelligence with machine learning in a manner that does not disturb the daily routines of the user by
\begin{itemize}
  \item Being conservative, see \cref{sec:interviewReports}, meaning that it is better not to act then to act wrongly
  \item Being reactive, meaning that the user should not notice any delay from the initiation of events to the execution
\end{itemize}

The setup and usage of the system should focus on fulfilling the following three criteria in relation to the user, found from \cref{sub:existingSystems}
\begin{itemize}
  \item Cheap
  \item User friendly
  \item Personalised
  \item Flexible
\end{itemize}

From section \cref{sub:Technologies} we have that the system must have sensor variety that can cover
\begin{itemize}
  \item Time
  \item Light intensity
  \item Location and spacial awareness
  \item Awareness of active devices
\end{itemize}

%mainfile: master.tex
\subsection{Problem Formulation}\label{problemFormulation}
In the previous chapters, the current state of home automation as well as the users thereof were analysed and discussed. During these discussions, some subjects and delimitations for the scope of this project were presented. Taking these into consideration, a concrete problem formulation was devised, presenting a solvable problem in a delimited domain of home automation. The problem formulation is as follows:

\emph{How is it possible to create a system, that uses several sensors to learn the residents’ usage patterns in their home, to intelligently control lights and appliances in order to minimise power consumption and automate reoccurring tasks, while minimising user interaction.}
