%mainfile: ../master.tex
\section{Conclusion \& Requirements}\label{sec:requirements}
This section will conclude on the analysis and from the knowledge gathered in \cref{sec:interviewReports} and \cref{sec:sota} about existing systems. The experience gathered from these existing solutions will define some requirements of this new system.

Based on these two reports \cref{sec:interviewReports} and previous analysis our system should be entirely wireless, be able to recognise whether an action is a temporary adjustment or a change in schedule, and minimise user interaction while still giving the user enough information to interact with the system appropriately. The user should be able to query the system of reasoning of actions taken.
\\\\
The system should focus on creating a smart home system, see \cref{sec:interviewReports}, that focus on lighting and electronic appliances. It should do so by utilising machine intelligence with machine learning in a manner that does not disturb the daily routines of the user by
\begin{itemize}
  \item Being conservative, see \cref{sec:interviewReports}, meaning that it is better not to act then to act wrongly
  \item Being reactive, meaning that the user should not notice any delay from the initiation of events to the execution
\end{itemize}
The setup and usage of the system should focus on fulfilling the following three criteria in relation to the user, found from \cref{sub:existingSystems}
\begin{itemize}
  \item Cheap
  \item User friendly
  \item Personalised
  \item Flexible
\end{itemize}
From section \cref{sub:Technologies} we have that the system must have sensor variety that can cover
\begin{itemize}
  \item Time
  \item Light intensity
  \item Location and spacial awareness
  \item Awareness of active devices
\end{itemize}
