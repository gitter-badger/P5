%mainfile: ../master.tex
\section{User Interviews}
\label{sec:interviewReports}
To get a better understanding of what users wants from a home automation system we looked at several different reports. One which did interviews regarding general home automation\cite{HAInterviews} and one which did interviews specifically regarding the smart thermostat Nest, with focus on machine intelligence integrated as part of a home automation solution\cite{AdaptiveInterviews}.
\\\\
The Nest specific report interviewed 23 people from 19 households between February and September 2012. 10 of them participated in a diary study as well. All interviews were conducted by phone except for one. The main focus were the participants' experience and understanding using the solution. 8 of the 10 people participating in a diary study, had obtained the solution fewer than three weeks before the study began\cite{AdaptiveInterviews}.

The report regarding general home automation performed 14 semi-structured interviews including 31 people across 14 households in the summer of 2010. All of the participants had been using their system for at least a year\cite{HAInterviews}.
\\\\
Two significant barriers to entry into home automation are inflexibility and poor manageability\cite{HAInterviews}. Inflexibility is concerned with the idea that installing a home automation system, requires doing rewiring in walls or even remodelling parts of the house to accommodate more extensive home automation solutions. The rewiring is caused by individual modules having to connect to the central server. Because of this, users switched to a wireless solution\cite{HAInterviews}.

Manageability describes the maintainability of the system along with the ability
to change how the system behaves. It was noted in the general report that even though the participants were well-equipped to handle these complex home automation systems, they still struggled to manage the systems and this would be a problem for widespread adoption. Especially the initial period of using the system, proved to be a very iterative process, because when some features were actually implemented, it turned out the users did not want or need part of the functionality. Especially functions tied to a schedule were problematic because people realised their lives were not as structured as they had thought. One person said: \enquote{I'd wake up and music is playing in my bathroom \ldots I don't live that structured of a life \ldots and I'm not going in the shower every day at the same time.}. Together with this problem, users found that creating general rules were too difficult, it is the \enquote{challenge of interference in the presence of ambiguity}\cite{HAInterviews}. It is difficult for a system, almost impossible for a system without machine learning, to infer the situation based on hard rules if there is any ambiguity. The fact that a lot of iterations were required, also meant that the users needed to manage with the system often. Making a user interface for managing the system a necessity.
\\\\
The manageability problem as outlined above could be alleviated if a system incorporated machine intelligence or machine learning. Which is what the Nest thermostat did. The Nest report interviews users to understand how they interact and understand \emph{Smart Home} systems. The Nest system included an \emph{Auto-Schedule} functionality, that relies on the use of machine intelligence. The auto schedule functionality would try to learn the user's schedule and have the thermostat react according to that. The machine learning functionality however turned out to be, as one user described it, too \enquote{aggressive}\cite{AdaptiveInterviews}. E.g. when he changed the temperature once, the system would assume he would always want it like that and adjust the schedule. The version of Nest used in this case, had a difficult time distinguishing between routine behaviour and temporary adjustments. Nest did not solicit user input, because it would go against the fundamental idea that automation should require as little user interaction as possible. Nest users would have liked to be able to understand why Nest made the decisions it did, because without some knowledge about how the system made decisions, the users felt it were exceedingly difficult to work with the system. The machine intelligence part of the Nest system arguably failed, as multiple users either turned it off completely or only used it as part of the study.
\\\\
Based on these two reports and previous analysis, a constraint of our system is that it should be entirely wireless, a functional requirement the system should be able to recognise whether an action is a temporary adjustment or a permanent change in schedule. Another constraint is that the system should minimise solicitation of user interaction, and another functional requirement is the user should be able to query the system of reasoning of actions taken.
