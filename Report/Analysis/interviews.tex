%mainfile: ../master.tex
\section{User Interviews}
\label{sec:interviewReports}
To get a better understanding of what users wants from a home automation system we looked at several different reports. One \cite{HAInterviews} which did interviews regarding general home automation and one which did interviews specifically regarding the smart thermostat Nest\cite{AdaptiveInterviews}, with focus on machine intelligence integrated as part of a home automation solution. 
\\\\
%hvordan foregik interviewsne
\cite{AdaptiveInterviews} interviewed 23 people from nineteen households between February  and September 2012. Ten of them participated in a diary study as well, all interviews were conducted by phone except for one. Main focus were the participants experience and understanding using the solution. Of the ten participating in a diary study eight had obtained the solution fewer than three weeks before the study began. 
\cite{HAInterviews} performed fourteen semi-structured interviews including 31 people across fourteen households, in the summer of 2010. All the participants had lived with home automation for at least a year.
\\\\
\cite{HAInterviews} makes a distinction between two terms; Smart Home and Home Automation. A Smart Home implies that the home "adapts to the inhabitants" and Home Automation is grouping multiple actions together and binding them to one user action (such as turning every light on in a room using only one button).\ranote{Think its a relevant distinction, might be misplaced though.}
\\\\
Two significant barriers to entry into home automation are inflexibility and poor manageability\cite{HAInterviews}. Inflexibility is concerned with the idea that installing a home automation system requires changing wires often in the walls or even remodelling parts of the house to accommodate more extensive home automation solutions. Because of this having the individual modules connect to the, possible, central server is crucial. Because of this user's in \cite{HAInterviews} switched to a wireless solution.
Manageability describes the maintainability of the system along with the ability to change how the system behaves. \cite{HAInterviews} noted that even though their participants were well-equipped to handle these complex home automation systems they still struggled and this would be a problem for widespread adoption. Especially the initial period proved to be very iterative because when the systems, the users had bought, were actually implemented it turned out they didn't want or need part of the functionality. Especially functions tied to a schedule were problematic because people realised their lives weren't as structured as they had thought\footnote{One person said: "I'd wake up and music is playing in my bathroom(...)all these Jetson type things. And the challenge with that (...) I don't live that structured of a life(...)}. Together with this problem users found that creating general rules were difficult, it's the "challenge of interference in the presence of ambiguity" \cite{HAInterviews}. It is difficult for a system, almost impossible for a system without machine learning, to infer the situation based on hard rules if there is any ambiguity. The fact that a lot of iteration were required also meant that the user needed to interact with the system often. This means that the user interface were important.
\\\\
The manageability problem as outlined above could be alleviated if a system incorporated machine intelligence or machine learning. Which is what the Nest thermostat did, \cite{AdaptiveInterviews} interviews users to understand how users interact and understand Smart Home systems. The machine intelligence integrated in Nest were used for a function called "Auto-Schedule". Which would try to learn the user's schedule and have the thermostat react according to that. But the machine learning were, as one user described it\cite{AdaptiveInterviews}, too "aggressive". E.g. when he changed the temperature once, the system would assume he would always want it like that and adjust the schedule. Although, with the version of Nest they used, Nest also had a difficult time distinguishing between routine behaviour and temporary adjustments. Nest didn't solicit user input, because it would go against the fundamental idea that automation should require as little user interaction as possible. But  users would have liked to be able to understand why Nest made the decisions it did. Because without some knowledge about how the system made decisions the users felt it were exceedingly difficult to work with the system.
Even though the machine intelligence part of the Nest system explored in \cite{AdaptiveInterviews} arguably failed (as multiple users either turned it off completely or only used it as part of the study), it is easy to see that a stronger machine intelligence system could solve some of the problems presented in \cite{HAInterviews}. 

Based on these two reports and previous analysis our system should be entirely wireless, be able to recognise whether an action is a temporary adjustment or a change in schedule, and minimise user interaction while still giving the user enough information to interact with the system appropriately.
\label{int:requirement}
