%mainfile: ../../master.tex
\subsection{Existing Systems}
\label{sub:existingSystems}
In this section, some of the most popular home automation systems and their functionality will be discussed.

\subsubsection{Home Seer}
%source on first statement
Home Seer is one of the most popular systems on the market. It is a modular home automation system, where the users are able to control and monitor their homes via an application on their smart phones. The user buys a controller that acts as the central computer system and can after that buy a very large range of modules of different types. The system is divided into 17 main categories.\footnote{The categories will not be explained in detail here but will be used and explained more in detail in the design if they are used.}
\begin{itemize}
	\item Lighting
	\item Thermostats / Climate Control
	\item Door Locks
	\item Garage Doors
	\item Cameras
	\item Security Systems
	\item Appliances
	\item Sensors
	\item Water Management
	\item Shades / Blinds
	\item Voice, Telephony
	\item Audio / Video / Media
	\item Energy Management
	\item Weather
	\item Automobile
	\item Fitness / Wearables
	\item Pool / SPA Control
\end{itemize}%Source: http://www.homeseer.com/compatible-products.html
When the devices are connected to the controller, the system can be controlled by any computer on the local network using the software HS3/HS3PRO. The user is able to manage the controller to set-up, manage, and disable hardware in the system. What makes this system somewhat unique is that the system allows the user to program each of the units individually. Allowing the user to set-up automated events like turning on lights at sunset or locking the door and closing the garage door when the car leaves the driveway. The system can also be set up to be controlled and monitored via an application on their mobile device like seeing which lights are turned on and being able to turn them off remotely.%http://www.homeseer.com/guides/HomeSeer-QuickStart-Guide.pdf

\subsubsection{Control4}
Control4 is in many cases the more user-friendly version of Home Seer. An expert is assigned to the user and they work together to find out what the user's needs are and how they want their home to be automated. The system is a lot less modular than Home Seer but is customised specifically to the people using it, both in software and hardware. This also makes it possible for Control4 to focus more on the home \enquote{knowing} when to do something, and actively doing it, instead of relying on inputs from the user. This also means that the system will make less mistakes and give the user a perceived higher quality product, but at a much higher cost.%http://www.control4.com/

\subsubsection{Samsung: SmartThings}
Samsung: SmartThings is effectively a middle-ground between Home Seer and Control4. It tries to make a user friendly and partially customisable smart home but at a relatively low price. This is done using common patterns for people, like coffee brewing in the morning and washing of clothes before coming back home. The set-up is short and only consists of a few questions for the user to personalise the experience to some degree. This is not as personalised as Control4 and Home Seer can be and is limited by the use cases designed by Samsung, but is a great ammount cheaper than Control4 and much easier and more user friendly than Home Seer.

The system works as a conversation between the user and the system. Making the user trigger certain events like morning routines by sending the message \enquote{Good morning} and reporting with useful information and status' like \enquote{The weather forecast for today is \dots Setting the temperature to 24\degree and starting to brew your coffee.}

\subsubsection{Apple: HomeKit}
HomeKit is a framework for communicating with and controlling connected accessories in a user's home. %Directly copied from https://developer.apple.com/homekit/
This system is designed for superusers and programmers to use Apples voice activation and recogniser software Siri, in order to be able to control their Home Automation system via voice commands.

\subsubsection{Delimitation}
Home automation solution currently available in the market greatly covers several aspects that can be automated within the problem domain. However all of these solutions requires the user to set-up a (static) configuration or limit the automation on predefined use cases. In this respect there is much room for improvement.
%The different home automation systems which exist today are very capable in the areas of the home which can be automated. However, all of these systems are very static in their behaviour, that is, they all operate according to their predefined behaviours, unable to change their behaviour to meet the user's changing needs, unless explicitly reprogrammed to do so. Following this train of thought, a market for a self-learning home automation system should exist, since such a system should be able to lessen the toll on the user, or expert, by removing the need to continuously configure the system.
