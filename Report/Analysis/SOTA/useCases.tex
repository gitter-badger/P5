\section{Usecases of Home Automation}
\label{sec:Usecases of Home Automation}
There is a certain amount of the tasks, which has to be done on a daily basis. These could be trivial tasks such as turning off the light when leaving the bathroom or more advanced ones as cooking dinner or doing laundry. Regardless of what type of task it is desirable to automate it in manner to increase the quality of life.
\\\\
In this section, some of the major use cases regarding smart home automation will be discussed. There are several uses of home automation and the major ones are the following:

\begin{enumerate}
  \item Security
  \item Surveillance
  \item Light Automation
  \item Entertainment Automation
  \item Room Temperature Regulation
\end{enumerate}

\subsection{Security}
\label{sub:Security}
The first and foremost use case of automation are security aspects. A case of such security case could be to detect intrusion. Protection is good but prevention is better. Another case of the category could be detection of fire. Detecting fire at the right moment can be crucial that could save either life or property. Monitoring and reporting hazardous event to the central and thus receive help at the right time can save lives and by atomising this task there will be less to worry about and thus improve on the quality of life.

\subsection{Surveillance}
\label{sub:Surveillance}
Surveillance is another area that benefits from automation. Automation in this area boosts the security as well as optimise few things. This type of automation is concerned with making the owner able to keep track of who is at the door and thereby giving the user the choice of opening door. Another specific use case would be automated doorbells that has to ways of audio and one-way video communication. The mail carrier could thereby communicate with the owner, if they are not present.

\subsection{Lighting Automation}
\label{sub:Lighting Automation}
Keeping track of lights in a home is a very typical task on a daily basis. Automation in this area will not only increase the of comfort but also helping the environment by using reducing waste and increasing efficiency.  The system could detect the amount of light as well as monitoring sunrise and sunset and thereby turning off light where the light is not needed and save energy.

\subsection{Entertainment Automation}
\label{sub:Entertainment Automation}
A very different aspect is entertainment automation. A way to put several entertainment devices together such that they can start to interact with each other and in that way increase the convenience to use a certain products. An example could be that the theatre system could communicate with the lights in the room as well as the climate system and thereby create a theatre environment on demand.

\subsection{Room Temperature Regulation}
\label{sub:Room Temperature Regulation}
A trivial task that is maintained on a daily basis is indoor heat control, cooling the house down when it gets too hot and heating it up again when it gets too cold. This process can be resource consuming and can very well be automated.
\\\\
Automation of heat control inside the home can contribute to solve environmental issues by automatically and intelligently monitoring and regulating the room temperature. Monitoring environmental influences, such as the weather and users preferred temperature the task could be optimised to save energy and improve the life quality.  The further the waste reduction the more environmentally friendly the system gets.
