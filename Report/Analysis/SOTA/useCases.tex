%mainfile: ../../master.tex
\subsection{Areas of Home Automation}
\label{sec:Areas of Home Automation}
There is a certain amount of tasks, which has to be done on a daily basis in a typical household. These could be trivial tasks such as turning off the light when leaving the bathroom or more advanced ones as cooking dinner or doing laundry. Regardless of what type of task, it is desirable to automate it in manner to increase the quality of life.
\\\\
In this section, some of the major areas which can be automated in a home will be discussed. There are several uses of home automation and the major ones are the following:

\begin{enumerate}
  \item Security
  \item Surveillance
  \item Light Automation
  \item Entertainment Automation
  \item Room Temperature Regulation
\end{enumerate}
These will be expanded on below.

\subsubsection{Security}
\label{sub:Security}
One area of automation is security. A specific area of security automation could be detection of fire. Detecting a fire at the right moment is crucial as it could save either life or property. Monitoring and reporting hazardous events to a central in order to receive help in sufficient time automatically will mean there will be less to worry about and thus will improve on the quality of life.

\subsubsection{Surveillance}
\label{sub:Surveillance}
Surveillance is another area that benefits from automation. Automation in this area boosts the security as well as optimising a few things. This type of automation is concerned with making the owner able to keep track of, for example, who is at the door and thereby giving the user the choice of opening door. Another specific area would be automated doorbells that have two ways of audio and one-way video communication. The mail carrier could thereby communicate with the owner, if they are not present, by a remote connection.

\subsubsection{Lighting Automation}
\label{sub:Lighting Automation}
Keeping track of lights in a home is a very typical task which has to be done daily. Automation in this area will not only increase the comfort of the user, but also help the environment by reducing waste and increasing efficiency. The system could detect the amount of light as well as monitoring sunrise and sunset and thereby turning off light, when the light is not needed and thereby save energy.

\subsubsection{Entertainment Automation} \kanote{lav om til appliances, dvs. basically alt der kommer til at stå på standby rundt omkring i huset. Merge med lighting?}
\label{sub:Entertainment Automation}
Another, slightly different, area is entertainment automation. A way to put several entertainment devices together such that they can interact with each other and in that way increase the convenience in using certain products. An example could be that the home theatre system interacts with the lights in the room as well as the climate system and thereby can create a theatre environment on demand.

\subsubsection{Room Temperature Regulation}
\label{sub:Room Temperature Regulation}
A trivial task that is maintained on a daily basis is indoor heat control; cooling the house down when it gets too hot and heating it up again when it gets too cold. Automation of heat control inside the home can contribute to solving environmental issues by automatically and intelligently monitoring and regulating the room temperature. By monitoring environmental influences, such as the weather and the user's preferred temperature, this task could be optimised to save energy and improve the life quality of the user.

\subsubsection{Delimitation}
Many different areas of the home can be automated by a home automation system. However, if a system is to be self-learning, there needs to be room for the system to make mistakes; thus some critical areas, such as security, should not be automated by such a system. Some of these areas also fit better with the idea of saving energy; lighting and room temperature regulation automation are perhaps the areas which have the biggest potential for saving energy. Therefore these areas are chosen as the main focus of this project.
