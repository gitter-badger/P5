\subsection{Device Communication Protocols}

All the devices installed in a automated home need some form of communication
method with other devices in order to send/receive data. For example, a light
bulb might need to know when to turn itself on. Another example could be
sending data collected from a temperature sensor to another device.

This section explores some protocols frequently used in the domain of home
automation. Each protocol has some characteristics that is outlined below.

\subsubsection{X10 \cite{wiki_x10}}

\begin{itemize}
\item Designed in 1975
\item Supports both powerline and wireless installation
\item Lacks encryption
\item Limited reliability
\end{itemize}


\subsubsection{Universal Powerline Bus \cite{wiki_upb}}

\begin{itemize}
\item Released in 1999
\item Powerline based
\item Limited bandwidth of 240 bits/s
\end{itemize}


\subsubsection{Insteon\cite{wiki_insteon}}

\begin{itemize}
\item Supports both powerline and wireless installation
\item Proprietary
\item Compatible with X10
\item Uses powerline as backup in case of wireless interference
\item Theoretical bandwidth of 2880 bits/s
\end{itemize}


\subsubsection{Z-wave \cite{wiki_zwave}}

\begin{itemize}
\item Wireless
\item Proprietary
\item Theoretical bandwidth of 100 kilobits/s
\item Line of sight range is approx. 100 meters
\item Designed for low power consumption
\end{itemize}


\subsubsection{ZigBee \cite{wiki_zigbee}}

\begin{itemize}
\item Originally designed in 1998, standardised in 2003 and revised in 2006
\item Based on IEEE 802.15.4 open specification
\item Wireless
\item Line of sight range is 10-100 meters
\item Theoretical bandwidth of 250 kilobits/s
\item Designed for low power consumption
\end{itemize}

\subsubsection{WiFi \cite{wiki_wifi}}

\begin{itemize}
\item Wireless
\item Designed for high-bandwidth rather than low-power consumption
\item Open standard
\end{itemize}


\subsubsection{Bluetooth \cite{wiki_bluetooth}}

\begin{itemize}
\item Wireless
\item Designed for short-range, low-power consumption
\item Ranges vary from 1 meter to 10 meters to 100 meters for industrial use
\end{itemize}

\subsubsection{Ethernet \cite{wiki_ethernet}}

\begin{itemize}
\item Wired connection
\item High bandwidth - 100 gigabit/s
\item Low latency
\end{itemize}

Some observations can be made about the listed protocols. Considering the
use-site of the devices, homes, wireless protocols seem most fitting. Protocols
designed for low power consumption are a must when the devices using the
protocol are battery-based. However they do not allow for high data transfers
and a high range.