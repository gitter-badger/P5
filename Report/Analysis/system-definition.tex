\section{System Definition}

This section formally describes in which context the system is to be used, its
expected functions, the philosophy of the system, the conditions in which the
system is used, the technological aspect of the system, and the objects the
system should recognise and process.

\subsection{FACTOR}

The definition of the system can be decomposed into functionality, application
domain, conditions, technology, objects, and responsibility. This forms the
acronym FACTOR as introduced in \cite{mathiassen2001objektorienteret}.

\subsubsection{Functionality}

The system should be able to learn usage patterns of users, and autonomously
perform these actions.

\subsubsection{Application Domain}

The system should be used in homes where one or more users live. Each user may
have different usage patterns and sometimes irregular actions not part of their
daily routine.

\subsubsection{Conditions}

The system should adopt its actions to the wide-ranging usage patterns of a home
of multiple users.

\subsubsection{Technology}

The system should have small devices with accommodating sensors and actuators scattered strategically around the home.
These devices should be energy efficient in their use. Another computional
device should analyse sensor data collected by the small devices, and based on
the analysis, propose actions to be made autonomously by the system.

\subsubsection{Objects}

The system should recognise the users' patterns and their interaction with the
rest of the home.

\subsubsection{Responsibility}

The system should be as transparent to the user as possible to minimise friction
between the system and the user. This means that the user should not be
frustrated about the actions of the system. The system should be conservative
about its actions as to not perform actions the user would otherwise not have made.

\subsection{Definition}

The system should ease the lives of its users by performing actions autonomously
on behalf of them. These actions should be learned by sensing the users' usage
patterns throughout the day. As the user may perform irregular actions, the
system should be conservative about performing actions. The systems should
therefore only perform actions when it has a understanding of the action
pattern. The system should be as transparent to the user as possible. The system
should be an aid, not an element of frustration. The system should use small
computational devices, that conserve energy consumption, to sense the usage
patterns of the users. The data sensed should be communicated with another
computational device that analyses it, to further improve the knowledge of the
users' usage patterns, thereby improving the quality of actions the system can perform.