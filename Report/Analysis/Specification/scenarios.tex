%mainfile: ../../master.tex
\subsection{User scenarios}\label{sub:userscenarios}

Below are possible narratives of branches derived from the use cases, as described in the \cref{sub:usecases}.

\textit{A user is reading a book, the light intensity of the sun is low now due the time of the day, and the user is having trouble reading. The system has learned that at this light intensity when the user is situated at the couch with TV not running, the user now wants the light to be on.}

\textit{A user is home alone in his/her living room. Now leaving a room to go the toilet but does not turn of the light in the living room, while on the toilet the user closes the door and does not observe the light in the living room being turned off by the system to conserve/save energy.}

\textit{A user just left home for a vacation, the user is leaves a few lights on due to stress, the system recognises this and switches the lights off. The user has not been home for 24 hours, the system now initiates in doing short cycles of simulating normal behaviour of the user in some rooms to seed the impression of someone being home, to proactively prevent attracting interest from any observing burglars.}

\textit{In a financial report an organisation recognises a substantial amount of funds is used on lighting the facility. The system generates a report on the lights around the facility, effective lighting hours. Which the supervisors can conclude on to argue were to invest on more energy efficient lighting. Further the system reports on broken or not functioning lighting.}

%Just for fun ? Many users is situated in the living room and the volume(db) of the stereo is really high, so users is shouting to hear each other and this occurs over prolonged duration, the system intellingently lowers the volume without the user noticing. And at 1am/pm(night) the user normally goes to sleep but not today, but because of neighbors the system lowers the volume further.

