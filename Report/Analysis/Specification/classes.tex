%mainfile: ../../master.tex
\subsection{Classes}

From the system definition and "rich picture" of the problem domain that the system will be operating on/in, we can extract some classes. 

\begin{quote}
A description of a collection of objects with the same structure, behaviour and attributes
\end{quote}
as described in \cite{OOAD}

These classes will be beneficial when figuring out what we need to keep track of in the problem domain, or what is relevant in our model of the problem domain.

This system should be able to keep an always up to date model of the problem domain, \cref{lis:classes} summarises some candicates for classes that the system use to structure this model.

\begin{figure}
  \label{lis:classes}
  \begin{itemize}
    \item Person - The user of the system
    \item Action - An action the system should recognise as feedback
    \item Pattern - The usage patter of a user, what is typical behaviour
    \item Mistake - System did something wrong by the user
    \item Sensor - The observational member of the system
    \item Switch - A feedback source
    \item Lamp - The light source on which system action
    \item Light intensity - Luminance
    \item Room - What room is the user in and its relations
    \item Premise - The user home enviroment
    \item Time - At what time, a pattern property
  \end{itemize}
  \caption{Candicates of classes}
\end{figure}
