%mainfile: master.tex
\chapter{Introduction}\label{part:introduction}

Home automation is a growing industry and is slowly gaining ground with the users. More and more homes are getting automated, but it is still a rather unexplored technology and far from every home has it. Although the technology is progressing, home automation has been quite a long time under way, compared to nearly all other electronics- and information technologies. The concept was first introduced in the 1930s during many of the fairs around the world, especially the 1933 \enquote{Homes of Tomorrow Exhibition} at the \enquote{A Century of Progress International Exposition} in 1933 to 1934 in Chicago, USA.

Even though the technology has been in development for over 80 years, it is still a rarity to own, or for that matter, even know someone who owns, a home automation system. Compared to mobile phones, first commonly theorised in 1948 in the science fiction novel \enquote{Space Cadet} by Robert Heinlein, and now with modern society assuming people to own one, home automation seems to be lagging behind. So how come that this is the case? How come that this technology seems so hard for companies to make popular?

This project was devised to find out which features are good and which are bad in current home automation technologies, and, with these features in mind, provide an innovative solution to the problem of popularising home automation.

%Comfort is also a requirement beside minimising power usage.
%Elders in need of (hjemmehjælp)
